
%%% Local Variables:
%%% mode: latex
%%% TeX-master: t
%%% End:

\chapter{绪论}
\label{cha1}

\section{引言}

 视网膜是眼部结构的内层,是一层透明的薄膜,由于色素上皮细胞和脉络膜的关系,眼底呈现均匀的橘红色~\cite{retinal}。视网膜负责感光成像,就如一架照相机里的感光底片,同时具有初步的信息处理功能。人眼看东西时,物体发出的光线经过折光系统在视网膜上成像,视网膜的感光细胞会将感到的光转变为神经信号,并传递到大脑,大脑各部分进行工作,建立起看到的物体的概念。
 
视网膜的中心是视神经盘,由于这里没有感光细胞,因此也被称为盲点。而动脉和静脉血管构成了复杂的血管网络,动脉看起来更明亮些,因为动脉负责运输氧气丰富的血液给身体的各个器官,静脉血管的血液含氧量较低,颜色更加深一些。同时动脉血管比其周围的静脉血管要细一些,动脉中的中心反射现象比静脉中的更明显。视网膜中的血管形态几何结构符合某些物理性质的结构准则,其唯一性的结构构造可以用来做生物识别方面的研究,在某些研究里作为初步的识别步骤。图 ~\ref{hrf1}中所示为一幅典型的视网膜眼底图像及其对应的血管结构图~\footnote{https://www5.cs.fau.de/research/data/fundus-images/}。
 \begin{figure}[ht!]
    \centering
 \begin{minipage}{0.45\linewidth}
  \includegraphics[width=\linewidth]{hrf1.png}
\end{minipage}
\begin{minipage}{0.45\linewidth}
  \includegraphics[width=\linewidth]{hrf2.png}
\end{minipage}
\caption{一幅典型的视网膜眼底图像及血管骨架图}
  \label{hrf1}
 \end{figure}
 
正因为视网膜在人体健康方面的重要作用,对于它的相关方面的研究具有重要的意义。研究人员通过OCT图像采集等方法将不同个体的、不同时间、不同模态的图像采集下来,这些视网膜血管图像包含珍贵的局部和时间信息,因此广泛用于分割、识别、配准等目的的研究。
  
视网膜图像分析在生物图像分析及基于计算机的疾病诊断方面吸引了许多学者和研究人员的关注,并得到了较大的发展。这里我们着重介绍视网膜图像配准在医学方面的重要作用。现今,视网膜图像配准技术越来越多的应用在一些眼科方面的疾病的诊断和治疗方面:对用激光治疗前后的视网膜图像进行配准以找到因激光造成的疤痕和烧伤;对同一个患者不同时间采集的视网膜图像进行配准以追踪病变的过程,如糖尿病视网膜病变、老年黄斑变性、青光眼、巨细胞病毒视网膜病变等疾病;实时的配准方法还可以在激光手术中用作眼科医生的基本工具使用。图 ~\ref{bad_retinal}中所示为糖尿病视网膜病变和青光眼的视网膜示例图像,可以观察视网膜已经发生了很明显的病变。
  \begin{figure}[ht!]
     \centering
 \begin{minipage}{0.45\linewidth}
  \includegraphics[width=\linewidth]{bad_retinal1.png}
\end{minipage}
\begin{minipage}{0.45\linewidth}
  \includegraphics[width=\linewidth]{bad_retinal2.png}
\end{minipage}
  \caption{糖尿病和青光眼病变的视网膜图像}
    \label{bad_retinal}
 \end{figure}
 
视网膜图像配准是一个极具挑战性的课题,这主要体现在以下几个方面~\cite{can2002}:
 \begin{enumerate}
\item 视网膜的表面是弯曲的,近乎球型,配准的变换模型需要将这点考虑进去。
\item 图像采集时的照明,有时过于刺目或者暗淡,造成视网膜的某一部分在不同的采集情况下有不同的灰度性质。而采集过程中产生的噪声也对特征提取等配准过程造成影响。如图 ~\ref{bad_retinal2}中所示,由于照明的强弱造成的视网膜采集图像的模糊、阴暗的情况。
\item 某些情况下,视网膜图像的大部分区域结构不够清晰,即血管结构特征不明显,主要的特征甚至只能观察到视神经盘。
\item 因为疾病或者年龄的影响,视网膜结构部分区域发生了较大的改变。
\item 视网膜血管在某些情况下可能只有2到3个像素的宽度,这就对配准的精度提出了更高的要求。
 \end{enumerate}
   \begin{figure}[ht!]
   \centering
 \begin{minipage}{0.45\linewidth}
  \centering
  \includegraphics[width=\linewidth]{bad_retinal3.png}
\end{minipage}
\begin{minipage}{0.45\linewidth}
  \centering
  \includegraphics[width=\linewidth]{bad_retinal4.png}
\end{minipage}
  \caption{质量较差的视网膜采集图像}
    \label{bad_retinal2}
 \end{figure}
 
 考虑到上述的问题,研究人员们针对不同方面进行了研究,我们在本文中提出的基于环结构的视网膜血管图像配准算法,有效利用了多尺度与多环的特征,并结合从局部到全局的配准策略,应用于视网膜图像数据集中,取得了良好的效果。
 

\section{图像配准基本概念及方法}
\subsection{配准基本概念和流程}

图像配准(Image registration)指的是将不同采集设备、不同时间或不同条件(气候、照明条件、视角等)下获取的同属一个场景的两幅乃至多幅图像进行匹配、叠加的过程~\cite{brown},它几何地将两幅图像进行对齐。通常,我们将在配准过程中不会发生改变,用作结果参考的图像称作参考图像,另一幅需要进行参数变换的图像称作待配准图像。图像配准广泛应用于计算机视觉,医学图像成像,生物成像和脑映射等领域。
 
图像配准的主要流程通常包括以下四步~\cite{zito}:
\begin{enumerate}
\item 特征检测。我们通常挑选一些具有显著特色的物体,包括某些区域、边界、轮廓、线的交叉等作为特征。作为进一步的处理,这些特征可以被表示为它们的点表达式,如使用重心来表示物体,线的末端点来表示直线等,这些点常被称作控制点(CPs)。通过上述方法对参考图像和待配准图像提取特征可得到相应控制点(特征点)。
\item 特征匹配。在这一步,参考图像和待配准图像的特征的对应关系会被检测出来。我们通过各种各样的特征描述符和相似性度量方法,利用特征中的空间关系,可以得到匹配的特征点对。
\item 变换模型估计。在这一步,会选择出合适类型的映射函数(变换模型)并求解出空间坐标变换参数,用于将参考图像和待配准图像进行配准。而该变换参数是通过上步求得的对应特征点对计算得到的。
\item 图像重采样和变换。待配准图像根据变换模型进行变换以完成最后的配准。
\end{enumerate}

\subsection{配准方法}
目前我们广泛使用的主要有两种配准方法~\cite{franc}。
\subsubsection{ 基于灰度信息的配准方法}

基于灰度的配准方法不需要预先进行提取特征等预处理工作,而是获取整幅图像的灰度信息进行统计分析,以此为依据,建立起参考图像与待配准图像间的相似性度量,然后通过寻找使得相似性度量得到最优值的空间变换模型参数,以完成配准。目前的基于图像灰度信息的方法主要有互相关,相位相关和互信息等方法。

基于灰度信息的配准方法需要得到整个图像的灰度信息以完成配准,并且为了找到最优的相似性度量值需要进行大量的运算,这大大提高了计算的复杂度。同时,在图像质量较差或两幅图像重叠覆盖的区域较小的情况下,基于灰度的配准方法可能无法成功的进行图像配准。

\subsubsection{ 基于特征的配准方法}

基于特征的配准方法首先需要对图像进行某些预处理操作,如分割和特征提取,然后利用得到的特征进行相似性度量已完成对应特征的匹配,并建立起两幅图像间的空间映射关系,最后完成配准。

这些特征通常为图像中显著的区域(如湖泊~\cite{helm},田地,建筑~\cite{hsieh})、线条(如区域边缘,物体轮廓~\cite{dai},海岸线,路径~\cite{li})、点(如区域拐点,线的交叉点~\cite{stock},小波变换的局部极值点~\cite{serief})等。理想的配准特征应该是独特的、稳定的、广泛分布于图像并且能够在图像中有效检测到的。特征之间的相似性由特征提取的准确性决定,这也要求检测到的特征之间的相同元素应尽可能的高,同时尽量不受到图像几何形变,附加噪声和扫描场景的影响。

与基于灰度信息的配准方法相比,基于特征的配准方法表达信息在更高的层次,可以极大减少计算量、提高算法的效率,同时对图像灰度变化有一定的鲁棒性。但是,由于该算法通常只采用图像的部分特征信息,所以它对于特征提取、特征匹配的精度及准确性要求较高,同时对错误也非常敏感。

\subsection{特征匹配}
在我们通过以上两种方法得到灰度信息、区域、点等特征后,需要进行特征的匹配以得到对应的特征对用于求取变换参数,而特征匹配是通过相似性度量完成的。

对基于灰度的配准特征匹配来说,相似性策略又分为基于灰度差值,灰度互相关和信息论的度量方法。灰度间的差值度量通常依据于像素灰度差的平方和(SSD)或归一化值~\cite{ashb},SSD值越低,说明两幅图像的相似性越好。互相关方法则根据Pearson的互相关系数或互相关比率,作为图像的相似性度量。而基于信息论的相似性度量则大部分依据于互信息(MI)~\cite{viola}或由其衍生的方法。

对于基于特征的配准方法来说,特征匹配的方法主要有基于空间关系的方法,基于不变描述符的方法等。其中基于空间关系的方法主要通过计算控制点间的距离信息和它们的空间分布信息。如~\cite{borge}在极小化其间的广义距离来进行配准基础上提出了结合均方根平均值,对连续的变换距离进行更好的度量。特征匹配的另一种选择,是对特征的描述子进行估计匹配,即基于不变描述符的方法。这种方法是将特征用某种描述方法表达,然后计算两幅图像中特征的描述的相似程度决定是否为对应特征,其中特征的描述子需要满足以下的几个条件:不变性、唯一性和稳定性等。最简单的特征描述子是图像的灰度值,其他还有形状,分叉点角度,矩不变量等。

\subsection{变换模型}
在找到控制点的对应关系后,需要进行变换模型的选择及参数估计。一般来说,变换模型的类型取决于待配准图像假定的几何形变,取决于图像获取的方法(如图像扫描仪可能造成的失真和误差),及要求的配准结果的精度。

根据所使用的图像数据的数量,可以将变换模型分为两种类型:全局模型,这种方法会使用所有的控制点来估计一系列的映射函数参数,这些参数对整张图像都是有效的;另一种是局部映射模型(或弹性变换模型),这种方法将图像看作由一系列patches组成,由每个局部区域分别决定映射模型的参数。

常用的全局映射模型有相似性变换模型、仿射变换模型、投影变换模型及二阶或高阶多项式变换等。几种变换模型各有特点,需要根据实际需要进行选择。在实际操作中,高阶多项式的使用较少,因为它们在待配准图像进行变换时经常会在远离控制点的区域造成没必要的扭曲失真现象。

通常控制点的数量要比变换模型要求的最低参考点数量要多。变换模型的参数估计然后由最小二乘法~\cite{ume}计算完成,该算法可以保证控制点造成的计算误差最低,但可能造成最后的控制点不是准确地一对一的。

局部的映射方法需要了解局部区域的可用信息,在某些场景下局部变换比起全局变换更加灵活,比如处理局部特征的大的扭曲等。

\subsection{精度估计}
对于配准结果,我们需要一种评价方法来对配准结果进行准确性评估,然而在图像配准领域,目前并没有一个统一的评估方法。目前最主要的评估方法有两种:
\begin{enumerate}
\item 专家主观评估:由医务工作者或研究人员凭借自己的经验来判断配准的准确性。配准结果的等级可分为四类:
\begin{enumerate}
\item 好(Good):配准结果较好,可做实际研究使用,误差在一个像素以内。
\item 可接受(Accepted):配准结果不如(a),但也可以接受,误差大于一个像素。
\item 不可接受(Not Accepted):配准结果不够好,无法用于研究使用,误差在一个像素到三个像素之间。
\item 差(Bad):配准结果无法使用,误差较大,度量值大于三个像素。
\end{enumerate}
然而这种方法不仅耗时耗力,效率较低,同时用人眼观察存在着较多的主观因素,并不是一种十分科学的方法。
\item 客观评估:即根据配准结果的数据进行准确性的评估。客观评估以数据为依据,最大限度的去除了人为因素的干扰。
\end{enumerate}
常见的客观准确性评估方法有以下几种~\cite{ume}:

\textbf{匹配误差}:匹配误差即计算在对应控制点对中,错误匹配的对数。在配准的特征匹配阶段,这是需要极力避免的错误,因为错误的匹配通常会造成配准失败。为避免错误的匹配,可以采用匹配算法和交叉验证的方法进行匹配。匹配算法可以单独使用,也可以同时使用两种不同的匹配算法,通过一致性检验来鉴别错误的匹配,只有那些被两种算法全都鉴别为有效的控制点对才会保留下来用于后续的配准。在没有可靠的匹配算法的情况下,可以使用交叉验证的方法来剔除错误的控制点对。交叉验证方法的具体过程如下:在每一轮从控制点对集中去除一对控制点并计算映射模型参数,然后检查在这种情况下该剔除点对的映射情况,若这对控制点间的偏移量小于阈值,则认为这对点是有效的。

\textbf{对齐误差}:它指的是我们用来配准的映射模型与实际的图像间的几何形变的差距。实际中对齐误差不可避免,这是由于我们选择的映射模型与图像实际的几何形变可能不一致,而计算的模型参数也可能不准确。常见的对齐误差估计方法有:计算对应控制点的均方误差(CPE)估计、测试点误差(TPE)估计等。在对齐误差估计方面,同样可以采用一致性检验的方法。

\section{视网膜配准国内外研究现状}
视网膜配准作为眼科疾病的研究、预防、治疗的重要手段,获得了众多研究人员的关注。其中视网膜配准方法也主要分为两类:基于灰度信息的配准方法与基于特征的配准方法。

在基于灰度信息的配准方法方面,互相关方法已经应用于单模态的眼科图像配准~\cite{cide},互信息~\cite{skokan}也成功应用于多模态的视网膜图像配准。然而在两幅图像之间的覆盖率较低的情况下,相似性矩阵的计算通常会被非覆盖区域的错误信息误导,因此基于灰度信息的方法通常无法成功的完成图像的配准工作。为克服这个困难,人们提出了基于感兴趣区域(ROI)的方法来计算相似性矩阵,以提高相似性矩阵的正确率,增强配准结果准确率。然而,除此之外,基于灰度信息的方法在一些图像质量较差如照明发生变化的情况,也不能进行很好的处理。

一些研究表明基于特征的方法更加适合于视网膜图像配准。基于特征的方法过程通常为:提取特征,由相似性度量找到对应的特征,再通过合适的变换方法,得到变换参数,最后完成配准。通常视网膜配准中使用的特征有血管网络,分叉点,视神经盘,视网膜中央凹等特征。

常用的不依赖于血管特征的一般性的视网膜图像配准方法有:Lowe等利用尺度不变特征变换(SIFT)~\cite{lowe},一个用于提取有特色的不变特征的算法,完成了视网膜配准。在算法中指出,SIFT特征对图像缩放和旋转是保持不变的,并且对于变换造成的扭曲形变,视角的改变,照明的改变及噪声的存在都具有足够的鲁棒性。然而SIFT特征仅被设计用来进行单模态配准,它的尺度不变性可能造成在进行高阶变换时,没有足够的控制点对。另一个得到广泛使用的特征是加速稳健特征(SURF)~\cite{bay},SURF依据于哈尔小波变换,充分利用了SIFT的长处,比SIFT特征速度更快、对图像变换的鲁棒性更高。

依据于视网膜常见的血管等结构特征,人们做出了如下的研究:Peli等~\cite{peli}在文中提出了基于视网膜血管和视神经盘特征的配准方法。Zana和Klein~\cite{zana}使用分叉点及其周围的血管方向用于多模态配准。很多基于特征的方法使用分叉点作为配准特征,因为它是血管的一个显著特征。Can和Stewart等~\cite{can2002}提出了一个分级的基于landmark特征(由分叉点,血管周围分支的方向,血管宽度组成)的配准算法,其参数估计阶段分为三级:初始匹配集由基于零阶的平移变换和相似性权重直方图估计;然后采用仿射变换和最小均方误差对减少的匹配集进行再次估计;最后,采用12个参数的高阶变换得到最终的变换参数和对应控制点。Can和Stewart等提出的算法可以有效避免不匹配的对应特征和特征的错配现象,后在此基础上,Stewart等又进一步提出了Dual-Bootstrap迭代最近点(Dual-Bootstrap ICP)算法~\cite{stewart}。通过采用从简单到复杂的变换模型,将bootstrap区域从局部到全局扩展,在模型阶数和区域大小两个方面逐步提高,最终完成整幅视网膜图像的配准,实验结果表明,最后的配准结果精度在一个像素之内。后续Tsai等~\cite{tsai}基于GDB-ICP算法,实现了视网膜多模态荧光素血管造影图像的配准。此外,Chanwimaluang等~\cite{thi}提出了一种结合区域和特征的混合配准方法,其中使用血管树的分叉点或交叉点作为landmark特征。Perez-Rovira等~\cite{adria}提出了RERBEE算法,使用血管结构(分叉点和血管片段)特征进行配准。这些方法大都依赖于单个分叉点或交叉点的分支角度,因此特征准确度较低并且易造成错配,对于配准来说不够唯一和可靠。

上述的算法,都可以称作点匹配配准算法,其过程是找到两个点集的空间变换关系以完成对齐匹配。在基于特征的图像配准算法里,点集即通过特征提取得到的特征。典型的点匹配算法有迭代最近点(ICP)算法~\cite{besl}、鲁棒点匹配(RPM)算法~\cite{gold}等。与点匹配算法相比,结构匹配算法的特征更具唯一性,因此能有效避免错配的发生。如Chen等~\cite{chen}\cite{chen15}提出了一种bifurcation结构作为配准特征,bifurcation结构即由一个主分叉点和与它相连的三个分支点组成的结构,并使用归一化的分支角度,相邻的两个分叉点间的长度作为特征向量。Shen等~\cite{shen}\cite{shenben}在bifurcation结构上提出了局部精细配准的方法对bifurcation结构方法进行扩展。然而,在血管图太过细节的情况下,bifurcation结构对于配准来说仍然不够唯一和可靠,并且由于分割采用的为单一尺度,如果分割结果较差,对分割血管要求较高的bifurcation结构可能不会取得较好的配准结果。

考虑到上述情况,我们提出了环结构这一新的血管特征用来进行视网膜的图像配准。所谓环结构,是由视网膜的动脉和静脉的血管的分叉点、交叉点及相连接的血管组成的结构。与其他特征相比,环结构更加具有唯一性和可靠性,其对平移,旋转,缩放具有不变性,不仅可以用于图像配准,还可以用于视网膜识别和验证等方面。为了克服配准对于血管分割的依赖性,我们采用了多小波核和多尺度分级分割的方法以得到多个尺度的血管图,不同的尺度代表了视网膜血管分割的不同细节程度。然后我们采用提出的基于空间信息的深度优先搜索算法(SDFS)来提取环结构,并根据提取的拥有不同特征点数的三、四、五点环组成多环结构。根据对应的环结构-血管特征点,我们进行了初次的局部初始配准,同时为了处理远离环结构的区域可能出现的血管未对齐情况,我们又进行了二次全局配准。经过从局部到全局的两级配准策略,我们得到了完整并且稳定的环-血管-分叉点特征,并得到了最终的配准结果,最后,骨架化对齐误差估计(SAEM)被提出以得到最优的配准尺度和环结构,同时选择出最优的配准结果。图~\ref{framework}为我们提出的视网膜血管图像配准算法流程图。
\begin{figure}[ht!]
\centering
\includegraphics[width=0.95\linewidth]{framework}
\caption{基于多尺度和多环结构的从局部到全局的视网膜血管图像配准算法流程图}
\label{framework}
\end{figure}
 
 \section{主要工作内容及安排}
 本文的主要工作内容及安排如下:
 
第一章为绪论,主要介绍了研究的相关背景、课题来源,重点阐述了图像配准的基本概念和各个流程的研究内容,和国内外视网膜配准研究发展现状及本文的主要研究工作内容。
 
第二章主要介绍了视网膜图像的预处理过程,包括多尺度分割,连通区域标记去噪,填充孔洞和骨架化过程。同时提出了多尺度的概念,表明多尺度分割对于配准特征提取的重要作用。最后介绍了环结构的概念:环结构是由动脉和静脉的血管的分叉点、交叉点及相连接的血管组成的结构。阐述了图论中关于环的概念,并给出了寻找环结构就是寻找视网膜血管图的最小环基的推论。
 
第三章介绍了环结构检测步骤:分叉点与交叉点检测、滤除无效特征点,提取环结构。重点介绍了环结构提取算法:基于空间信息的深度优先搜索算法(SDFS)。最后介绍了环结构描述方法,及多环结构的概念及构造方法。
 
第四章描述了从局部到全局的配准策略。提出了对环结构进行扩展,构造环-血管特征作为局部初始配准变换模型的输入。介绍了多种几何变换模型,并提出选择相似性变换作为变换的模型。在局部初始配准的基础上,介绍了利用全局分叉点,组合成环-血管-分叉点特征,进行二次全局配准的策略。
 
第五章介绍了配准误差度量方法:骨架对齐误差度量(SAEM)算法。阐述了我们进行配准实验所用的数据集,并进行了变换模型、配准特征、配准方法等对比实验。
 
第六章对本文的主要工作进行了总结,并提出了对未来的展望。
