%%% Local Variables: 
%%% mode: latex
%%% TeX-master: t
%%% End: 

\chapter{总结与展望}
\label{cha6}

\section{总结}
视网膜图像配准作为医学图像处理的重要组成部分,广泛应用在眼科疾病的诊断和治疗方面,因而具有重要的研究意义。在本文中,我们提出了基于多尺度与多环特征,和从局部到全局的配准策略的视网膜血管图像配准算法。定性与定量的实验结果表明,我们的算法取得了较好的配准效果,与其他经典的结构特征算法相比,在配准成功率及配准误差方面具有较大优势。

本文的主要贡献如下:
\begin{enumerate}
\item 提出了一种新的视网膜配准结构特征:环结构。环结构是由动脉和静脉的血管的分叉点、交叉点及相连接的血管组成的结构。实验结果表明,与其他配准特征相比,环结构更具有唯一性和鲁棒性。同时,本文提出了基于空间信息的深度优先搜索算法(SDFS)来提取环结构,该算法可广泛应用于无向无权图的最小环基的搜索,实现多点环结构的提取,具有较强的实际意义。然后,我们将三、四、五点环结构进行多种组合,得到了多环结构特征,用于后续的配准。
\item 提出了从局部到全局的配准策略。首先,对环结构特征进行扩展,得到环-血管特征,作为变换模型的输入完成局部的初次配准;然后,由参考图像与变换后的待配准图像,找到图像全局范围内未配准的分叉点,与环-血管组成环-血管-分叉点特征,进行全局的二次配准。通过以上的配准策略,可以较好的提高最终配准结果的准确性。该配准策略还可以应用于其他配准特征,实验结果表明,对于提高其配准结果的准确率有较大作用。
\item 提出了配准结果度量方法:骨架化误差度量算法(SAEM),成功应用于对配准结果的评价和度量。同时,该算法通过与我们的多尺度、多环特征及从局部到全局的策略结合,在自动挑选出最优配准结果的基础上,得到了参考图像与待配准图像的最优分割尺度及环结构。
\end{enumerate}
同时,我们提出的特征和方法还可以应用于视网膜识别、验证,或其他视网膜相关的研究。


\section{展望}
本文中提出了基于多尺度与多环特征,和从局部到全局的配准策略的视网膜血管图像配准算法,在实验中取得了较好的效果,但在未来仍需要做一些工作以对算法进行改进:
\begin{enumerate}
\item 算法复杂度。对一幅参考图像和一幅待配准图像来说,各自最多存在14个骨架化分割结果,最多有7种环结构组合,因此在我们提出的配准方法中,最多共有$14\times14\times7$个局部-全局配准结果,我们会在这些配准结果中根据SAEM选出最优的配准结果。因此,对于算法整体来说,需要计算大量的配准结果,而其中很多配准结果都是错误的、无用的,如何在算法中提前去除这些无效的配准结果,降低算法的复杂度,是下一步需要研究的内容之一。
\item 在实验中,我们采用的是VARIA数据集,该数据集图片较少,分辨率较低。目前我们的配准算法的实验结果均基于此数据集。在后续的研究中,我们考虑采用图像数量更多、质量更高的视网膜数据集来进一步验证我们的视网膜血管图像配准算法对多类数据集的适用性,并进行算法的改进。同时,我们希望该算法能够应用到实际中,以帮助眼科医生等进行视网膜相关疾病的诊断和治疗。
\end{enumerate}

